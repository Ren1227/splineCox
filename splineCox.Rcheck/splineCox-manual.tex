\nonstopmode{}
\documentclass[a4paper]{book}
\usepackage[times,inconsolata,hyper]{Rd}
\usepackage{makeidx}
\usepackage[utf8]{inputenc} % @SET ENCODING@
% \usepackage{graphicx} % @USE GRAPHICX@
\makeindex{}
\begin{document}
\chapter*{}
\begin{center}
{\textbf{\huge Package `splineCox'}}
\par\bigskip{\large \today}
\end{center}
\inputencoding{utf8}
\ifthenelse{\boolean{Rd@use@hyper}}{\hypersetup{pdftitle = {splineCox: A Two-Stage Estimation Approach to Cox Regression Using M-Spline Function}}}{}
\begin{description}
\raggedright{}
\item[Type]\AsIs{Package}
\item[Title]\AsIs{A Two-Stage Estimation Approach to Cox Regression Using M-Spline
Function}
\item[Version]\AsIs{0.0.1}
\item[Author]\AsIs{Ren Teranishi}
\item[Maintainer]\AsIs{Ren Teranishi }\email{ren.teranishi1227@gmail.com}\AsIs{}
\item[Description]\AsIs{This package implements a two-stage estimation approach for Cox 
regression using five-parameter spline functions to model the baseline hazard. It allows 
for flexible hazard shapes and model selection based on log-likelihood criteria.}
\item[License]\AsIs{GPL (>= 3)}
\item[Encoding]\AsIs{UTF-8}
\item[LazyData]\AsIs{true}
\item[RoxygenNote]\AsIs{7.3.2}
\item[Roxygen]\AsIs{list(markdown = TRUE)}
\item[Imports]\AsIs{joint.Cox}
\item[Suggests]\AsIs{knitr, rmarkdown}
\item[VignetteBuilder]\AsIs{knitr}
\end{description}
\Rdcontents{\R{} topics documented:}
\inputencoding{utf8}
\HeaderA{splineCox.reg1}{Fitting the five-parameter spline Cox model giving a specified shape}{splineCox.reg1}
%
\begin{Description}\relax
\code{splineCox.reg1} estimates the parameters of a five-parameter spline Cox model based on a specified shape for the baseline hazard function.
The function calculates the estimates for the model parameters (beta) and the baseline hazard scale parameter (gamma), using non-linear optimization.
\end{Description}
%
\begin{Usage}
\begin{verbatim}
splineCox.reg1(
  t.event,
  event,
  Z,
  xi1 = min(t.event),
  xi3 = max(t.event),
  model = "constant",
  p0 = rep(0, 1 + ncol(as.matrix(Z)))
)
\end{verbatim}
\end{Usage}
%
\begin{Arguments}
\begin{ldescription}
\item[\code{t.event}] a vector for time-to-event

\item[\code{event}] a vector for event indicator (=1 event; =0 censoring)

\item[\code{Z}] a matrix for covariates; nrow(Z)=sample size, ncol(Z)=the number of covariates

\item[\code{xi1}] lower bound for the hazard function; the default is min(t.event)

\item[\code{xi3}] upper bound for the hazard function; the default is max(t.event)

\item[\code{model}] A character string specifying the shape of the baseline hazard function. Available options include:
"increase", "constant", "decrease", "unimodal1", "unimodal2", "unimodal3", "bathtub1", "bathtub2", "bathtub3".
Default is "constant"

\item[\code{p0}] Initial values to maximize the likelihood (1 + p parameters; baseline hazard scale parameter and p regression coefficients)
\end{ldescription}
\end{Arguments}
%
\begin{Value}
A list containing the following components:
\begin{ldescription}
\item[\code{model}] A character string indicating the shape of the baseline hazard function used.
\item[\code{parameter}] A numeric vector of the parameters defining the baseline hazard shape.
\item[\code{beta}] A named vector with the estimates, standard errors, and 95\% confidence intervals for the regression coefficients
\item[\code{gamma}] A named vector with the estimate, standard error, and 95\% confidence interval for the baseline hazard parameter
\item[\code{loglik}] A named vector containing the log-likelihood (\code{LogLikelihood}),
Akaike Information Criterion (\code{AIC}), and Bayesian Information
Criterion (\code{BIC})
\end{ldescription}
\end{Value}
%
\begin{Examples}
\begin{ExampleCode}
# Example data
library(joint.Cox)
data(dataOvarian)
t.event = dataOvarian$t.event
event = dataOvarian$event
Z = dataOvarian$CXCL12

reg1 <- splineCox.reg1(t.event, event, Z, model = "constant")
print(reg1)

\end{ExampleCode}
\end{Examples}
\inputencoding{utf8}
\HeaderA{splineCox.reg2}{Fitting the five-parameter spline Cox model with a specified shape, selecting the best fit}{splineCox.reg2}
%
\begin{Description}\relax
\code{splineCox.reg2} estimates the parameters of a five-parameter spline Cox model for multiple specified shapes
and selects the best fitting model based on the minimization of the log-likelihood function.
The function calculates the estimates for the model parameters (beta) and the baseline hazard scale parameter (gamma), using non-linear optimization.
\end{Description}
%
\begin{Usage}
\begin{verbatim}
splineCox.reg2(
  t.event,
  event,
  Z,
  xi1 = min(t.event),
  xi3 = max(t.event),
  model = names(shape.list),
  p0 = rep(0, 1 + ncol(as.matrix(Z)))
)
\end{verbatim}
\end{Usage}
%
\begin{Arguments}
\begin{ldescription}
\item[\code{t.event}] a vector for time-to-event

\item[\code{event}] a vector for event indicator (=1 event; =0 censoring)

\item[\code{Z}] a matrix for covariates; nrow(Z)=sample size, ncol(Z)=the number of covariates

\item[\code{xi1}] lower bound for the hazard function; the default is min(t.event)

\item[\code{xi3}] upper bound for the hazard function; the default is max(t.event)

\item[\code{model}] A character vector specifying which model shapes to consider for the baseline hazard.
Available options are:
"increase", "constant", "decrease", "unimodal1", "unimodal2", "unimodal3", "bathtub1", "bathtub2", "bathtub3".
Default is \code{names(shape.list)} which includes all available models.

\item[\code{p0}] Initial values to maximize the likelihood (1 + p parameters; baseline hazard scale parameter and p regression coefficients)
\end{ldescription}
\end{Arguments}
%
\begin{Value}
A list containing the following components:
\begin{ldescription}
\item[\code{model}] A character string indicating the shape of the baseline hazard function used.
\item[\code{parameter}] A numeric vector of the parameters defining the baseline hazard shape.
\item[\code{beta}] A named vector with the estimates, standard errors, and 95\% confidence intervals for the regression coefficients
\item[\code{gamma}] A named vector with the estimate, standard error, and 95\% confidence interval for the baseline hazard parameter
\item[\code{loglik}] A named vector containing the log-likelihood (\code{LogLikelihood}),
Akaike Information Criterion (\code{AIC}), and Bayesian Information
Criterion (\code{BIC}) for the best-fitting model
\item[\code{other\_models}] A data frame containing the log-likelihood (\code{LogLikelihood}) for all other evaluated models,
with model names as row names.
\end{ldescription}
\end{Value}
%
\begin{Examples}
\begin{ExampleCode}
# Example data
library(joint.Cox)
data(dataOvarian)
t.event = dataOvarian$t.event
event = dataOvarian$event
Z = dataOvarian$CXCL12

M = c("constant", "increase", "decrease")
reg2 <- splineCox.reg2(t.event, event, Z, model = M)
print(reg2)

\end{ExampleCode}
\end{Examples}
\printindex{}
\end{document}
